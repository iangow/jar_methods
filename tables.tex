\begin{table}[t]

\caption{Descriptive Statistics} \label{tab:desc}

\justify{
{\bf RESTATE} is a zero-one indicator for whether a sample firm made a restatement in a particular year.
{\bf SALARY} is the CEO's annual base salary (in millions of \$).  
{\bf EQUITY} is the fraction of a CEO's total pay that is equity-based compensation.  
{\bf BIG4} is a zero-one indicator for whether the firm uses a Big 4 auditor.
{\bf FINDIRECT} is the fraction of the board of directors with a professional finance or accounting background. 
{\bf INT} is a zero-one indicator for whether the firm does the majority of its business (revenue)outside the US.  
{\bf SEG} is the firm's number of 2-digit SIC business segments.\\}

\vglue 10pt
\begin{center}
\begin{tabular}{|l|c|}
\hline
             & {\bf Sample }   \\
{\bf Variable}  &   {\bf Mean }   \\ 
          &  (Std Err)   \\ \hline
\T {\bf RESTATE} &   0.099  \\
                           &   (0.30)  \\[.6em]
    {\bf SALARY} &   1.06  \\
                           &   (0.27)  \\[.6em]
    {\bf EQUITY} &   0.45  \\
                           &   (0.26)  \\[.6em]
    {\bf BIG4} &   0.75  \\
                           &   (0.43)  \\[.6em]
    {\bf FINDIRECT} &   0.09  \\
                           &   (0.08)  \\[.6em]
    {\bf SEG} &   1.49  \\
                           &   (0.50)  \\[.6em]
    {\bf INT} &   0.31  \\
                           &   (0.46)  \\[.6em]
\hline
\end{tabular}
\end{center}
\end{table}

\begin{table}[t]
\caption{Logit Regression Results} \label{tab:logit}

\justify{This table presents results from logistic regressions of {\bf RESTATE}, a zero-one indicator for whether the firm made a restatement in a particular year, on a proxy for managerial incentives and controls. 
The controls are:
{\bf SALARY} is the CEO's annual base salary (in millions of \$).  
{\bf EQUITY} is the fraction of a CEO's total pay that is equity-based compensation.  
{\bf BIG4} is a zero-one indicator for whether the firm uses a Big 4 auditor.
{\bf FINDIRECT} is the fraction of the board of directors with a professional finance or accounting background. 
{\bf INT} is a zero-one indicator for whether the firm does the majority of its business (revenue)outside the US.  
{\bf SEG} is the firm's number of 2-digit SIC business segments.\\}

\begin{center}
\begin{tabular}{|l|cc|cc|}
\hline
             & \multicolumn{2}{c|}{\bf Specification 1}  &  \multicolumn{2}{c|}{\bf Specification 2}   \\
 {\bf Coefficient on}  &  {\bf Coefficient} & {\bf Marginal Effect} & {\bf Coefficient} & {\bf Marginal Effect}  \\ 
          &  (Std Err)  & (Std Err) &  (Std Err)  & (Std Err) \\ \hline
\T {\bf Intercept} &   -2.278   &   & -3.498   &    \\
                   &   (0.141)   &   & (0.198)   &   \\[.6em]
 {\bf SALARY}    &   0.280    &  0.025  & 0.326    &  0.028   \\
                   &   (0.120)   &  (0.011) & (0.121)   &  (0.010) \\[.6em]
 {\bf EQUITY}  &   -0.504   &  -0.045 & -0.503   &  -0.043  \\
                   &   (0.130)   &  (0.011) & (0.131)   &  (0.011) \\[.6em]
 {\bf BIG4 }  &       &    &  0.135    & 0.011    \\
                   &      &   & (0.080)   & (0.006)  \\[.6em]
 {\bf FINDIRECT}  &       &    & -0.239    & -0.020     \\
                   &      &   & (0.408)   & (0.034)  \\[.6em]
 {\bf INT }  &       &    & 0.548  & 0.051    \\
                   &      &   & (0.069)   & (0.007)  \\[.6em]
 {\bf SEG}  &       &    & 0.578   &  0.049  \\
                   &      &   & (0.069)   & (0.006)   \\[.6em]
\hline
\end{tabular}
\end{center}
\end{table}

\begin{table}[t]
\caption{Logit GMM Estimates for the Strategic Auditor Model} \label{tab:gmm}

\justify{This table presents GMM estimates for the strategic auditor model of Section \ref{sec:struct}. The first (exactly-identified) specification (1)
uses the instruments: Constant, {\bf SALARY}, {\bf EQUITY}, {\bf INT},
{\bf SEG}, {\bf SALARY*SEG}, and {\bf SALARY*INT}. Specification (2)
is over-identified. It uses the same instruments as (1) 
plus: {\bf EQUITY*SEG}, {\bf EQUITY*INT}, {\bf SALARY}$^2$,
and  {\bf EQUITY}$^2$. Specification (3) uses the instruments in (1) 
and imposes the constraints $\theta_0/\theta_3 = \theta_1/\theta_4
= \theta_2/\theta_5$, $\theta_0/\theta_1=\theta_3/\theta_4$
and $\theta_1/\theta_3=\theta_1/\theta_2$.
The "True" values are those corresponding to
the parameters used to generate the data.\\[2em]}

\begin{center}
\begin{tabular}{|c| c | c | c | c |}
\hline
\T  &  {\bf Estimated}    & {\bf Estimated} & {\bf Estimated}  & \\ 
  & {\bf Coefficient} &  {\bf Coefficient} &  {\bf Coefficient} & {\bf True} \\ 
\B  & {\bf (Std Err)}  & {\bf (Std Err)}  & {\bf (Std Err)} & {\bf Coefficient} \\ \hline
\T $\theta_0= \dfrac{(1-v_0)a_0m_0}{(1-p_I)(p_0-v_0)r_0b_0}$ & -0.016 & -0.009 & 0.008 & 0.007\\[-.5em]
& (0.045) & (0.044) & (0.002)  & \\[1.5em]
$\theta_1=\dfrac{(1-v_0)a_1m_0}{(1-p_I)(p_0-v_0)r_0b_0}$  &  0.049 & 0.055 & 0.053 &  0.050\\[-.5em]
& (0.034) & (0.035) & (0.007)  & \\[1.5em]
$\theta_2=\dfrac{(1-v_0)a_2m_0}{(1-p_I)(p_0-v_0)r_0b_0}$  &  0.060   & 0.055 & 0.051 & 0.050\\[-.5em]
& (0.030) & (0.030) & (0.006) & \\[1.5em]
$\theta_3=\dfrac{(1-v_0)a_0m_1}{(1-p_I)(p_0-v_0)r_0b_0}$  &  0.025  & 0.018 & 0.002 & 0.002\\[-.5em]
& (0.042) & (0.042) & (0.001)  & \\[1.5em]
$\theta_4=\dfrac{(1-v_0)a_1m_1}{(1-p_I)(p_0-v_0)r_0b_0}$ &  0.014  & 0.010 & 0.012 & 0.011\\[-.5em]
& (0.032) & (0.032) & (0.003) & \\[1.5em]
$\theta_5=\dfrac{(1-v_0)a_2m_1}{(1-p_I)(p_0-v_0)r_0b_0}$ &  0.003  & 0.008 & .012 & 0.011\\[-.5em]
& (0.028) & (0.028) & (0.001) &  \\[1.5em]
\B$\theta_6=\dfrac{b_1}{b_0}$ &  0.570  & 0.566 & 0.576 & 0.500 \\[-.5em]
& (0.184) & (0.184) & (0.261) & \\[1.5em]
\hline
\end{tabular}
\end{center}
\vglue 10pt
Robust standard errors are reported for specifications (1) and (2).
The standard errors for specification (3) are based on 500 Monte Carlo
replications. The parameter values used to generate the data are: $a_0 = 0.5 , a_1 = 3.5 , a_2 = 3.5, 
m_0 = 7, m_1 = 1.5, b_0 = 20, b_1 = 10, p_0 =0.75,
v_0 = 0.05, p_I = 0.45$ and $r_0 = 60.$
\end{table}
 
