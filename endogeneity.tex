\section{Causal inference in accounting research}

\subsection{The role of theory in empirical accounting research}
The vast majority of empirical research papers in accounting do not rely on a formal theoretical model to motivate their hypotheses.
This is perhaps inevitable given the wide range of theories that accounting researchers study, and the inherent complexity of many of the treatment and outcome variables.
For example, \citet{Huang:2014cs} study the effect ``tone management'' on capital market outcomes.
Developing a formal theory of the relation between firm performance, managerial psychological states, and measures of tone would be a complex undertaking involving economics, psychology, and linguistics.
Building on such a (hypothetical) foundation to solve the complex game involving managers and capital markets would be extremely ambitious.
Instead, \citet{Huang:2014cs} does what almost all empirical research papers in accounting do and resorts to ``verbal theorizing'.'

If most papers do 

\subsection{Endogeneity}

\section{Methods for causal inference}
\newpage

\subsection{Natural experiments}
Natural experiments occur when observations are assigned by nature (or some other force outside the control of the researcher) to treatment and control groups in a way that is random or ``as if'' random \citep{Dunning:2012tt}. 
Truly (as if) random assignment to treatment and control provides a sound basis for causal inference, enhancing the appeal of natural experiments.
However, \citet{Dunning:2012tt} argues that this appeal ``may provoke conceptual stretching, in which an attractive label is applied to research designs that only implausibly meet the definitional features of the method'' \citep[p.3]{Dunning:2012tt}.

Such ``conceptual stretching'' is evident in accounting research.
Our survey of accounting research identified six papers that exploited either a ``natural experiment'' or ``exogenous shock'' to identify causal effects \citep{Lo:2013jk,Aier:2014ii,Kirk:2014gx,Houston:2014hv,Hail:2014fq}.
But closer examination suggests that most of these papers misapply the fundamental idea of natural experiments.

\cite{Aier:2014ii} exploit a 1991 Delaware court that ``expanded the scope of directors' fiduciary duties to include creditors when a Delaware incorporated firm is in the `vicinity of insolvency.'" as a ``natural experiment'' for the purpose of understanding the causal effect of debtholders' demand for conservatism (the treatment variable) on financial reporting conservatism (the outcome of interest). But it is completely unclear how this ``natural experiment'' sorted firms into differing levels of the treatment variable, let alone why this assignment is as-if random.

 \citet{Kirk:2014gx} ``exploit the natural experiment setting created by the exogenous shock of Reg FD to examine the effect of Reg FD on firms with an established IR [investor relations] program.'' 
Given that the treatment of interest in \citet{Kirk:2014gx} is the establishment of an IR program, only a event that randomly assigned firms to having or not having such a program would qualify as a natural experiment in that setting.

\cite{Houston:2014hv} analyzes ``whether the political connections of listed firms in the United States affect the cost and terms of loan contracts'.' They argue that ``the recent financial crisis can be viewed as a major exogenous shock, the effects of which may vary depending on whether the firm is politically connected.'' But this is not what is needed for a valid natural experiment. To see this, an analogy is perhaps helpful. Suppose we wanted to study the effects of smoking on life expectancy. A long-standing concern in studies of such effects is the existence of other differences in the lifestyles of smokers and non-smokers. A ``natural experiment'' analogous to that in \cite{Houston:2014hv} might be one that created gas leaks in the homes of smokers and non-smokers alike. Because gas leaks are likely to have more deleterious consequences on smokers (e.g., instant immolation when lighting a cigarette), the reasoning of \cite{Houston:2014hv} might suggest that gas leaks are a helpful ``exogenous shock,'' contrary to common sense.

A plausible explanation for the ease with which conceptual stretching has occurred derives from the ambiguity of the word ``exogenous,'' which not only denotes  ``of, relating to, or developing from external factors'' (Oxford Dictionary), but is also the antonym of ``endogenous.''
For example, the fact that Reg FD was perhaps not driven by factors related to firms' IR programs and firm-level capital market outcomes, it does not randomly assign firms into treatment and control groups and thus does not help resolve the endogeneity of IR programs with such capital market outcomes.

\subsection{Instrumental variables}
While much has been written on the issues with using instrumental variables (IV) as the basis for causal inference, as casual review of recent accounting research suggests that this has had little impact of accounting researchers' understanding of IV.

\citet{Larcker:2010fq}, after surveying the use of IV in accounting research, lamented that ``some researchers consider the choice of instrumental variables to be a purely statistical exercise with little real economic foundation'' and called for 
``accounting researchers \dots to be much more rigorous in selecting and justifying their instrumental variables.''

A cursory survey of more recent research suggests that little has changed since \citet{Larcker:2010fq} was published. \citet{Kim:2014fm} use director age as an instrument for director tenure in a study examining the effect of the latter on firm performance. 
``Importantly, research finds little or no association between age and performance \dots and a small negative association between age and executive functions \dots. 
Related to directors, Ferris et al. (2003) suggest that any positive effects from director experience increasing with age may be offset by older directors having less energy, posing a last-period risk, and viewing directorships as lucrative part-time jobs for their retirement years.'' 
But these arguments seem to invalidate age as an instrument for tenure. 
For age to be a valid instrument, there should be no unblocked causal path between age and performance except for the path via tenure.
 That possible positive effects may be offset by negative effects and thus detecting an association between age and performance is not a valid basis for claiming age to be a valid instrument.

\subsection{Regression discontinuity designs}

Likely effective in settings, but of limited applicability and care is needed to do it well and to interpret the results. Discuss examples from shareholder voting and the \$75 million threshold for SOX.

\subsection{Propensity score matching}

Does \emph{not} solve endogeneity, but many believe that it does.

\subsection{Overall assessment}

We agree that the revolution in econometric methods for causal inference has certainly been an exciting development.\footnote{Not sure about ``revolution''; need to look at \emph{Mostly Harmless Econometrics} to get the right term here.} However, we have two grave concerns with regard to this methods in accounting research. First, it is evident that accounting researchers understand these methods only poorly and frequently seek to apply them inappropriately. Second,  it is far from clear that these methods, properly applied, can support more than a tiny fraction of accounting research. Of more than a 100 papers published in the top three accounting journals in 2014, we identified just a small fraction that applied these approaches and the vast majority of these did so in ways that seem difficult to classify as appropriate.