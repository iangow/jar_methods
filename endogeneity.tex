\section{Causal inference in accounting research}

\subsection{Importance of causal inference}

\subsection{Endogeneity}


\section{Methods for causal inference}

\subsection{Natural experiments}
Natural experiments occur when observations are assigned by nature (or some other force outside the control of the researcher) to treatment and control groups in a way that is random or ``as if'' random \citep{Dunning:2012tt}.


\subsection{Instrumental variables}
While much has been written on the issues with using instrumental variables (IV) as the basis for causal inference, as casual review of recent accounting research suggests that this has had little impact of accounting researchers' understanding of IV.

\citet{Larcker:2010fq}, after surveying the use of IV in accounting research, lamented that ``some researchers consider the choice of instrumental variables to be a purely statistical exercise with little real economic foundation'' and called for 
``accounting researchers \dots to be much more rigorous in selecting and justifying their instrumental variables.''

A cursory survey of more recent research suggests that little has changed since \citet{Larcker:2010fq} was published. \citet{Kim:2014fm} use director age as an instrument for director tenure in a study examining the effect of the latter on firm performance. 
``Importantly, research finds little or no association between age and performance \dots and a small negative association between age and executive functions \dots. 
Related to directors, Ferris et al. (2003) suggest that any positive effects from director experience increasing with age may be offset by older directors having less energy, posing a last-period risk, and viewing directorships as lucrative part-time jobs for their retirement years.'' 
But these arguments seem to invalidate age as an instrument for tenure. 
For age to be a valid instrument, there should be no unblocked causal path between age and performance except for the path via tenure.
 That possible positive effects may be offset by negative effects and thus detecting an association between age and performance is not a valid basis for claiming age to be a valid instrument.

\subsection{Regression discontinuity designs}